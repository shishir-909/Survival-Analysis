% Options for packages loaded elsewhere
\PassOptionsToPackage{unicode}{hyperref}
\PassOptionsToPackage{hyphens}{url}
\PassOptionsToPackage{dvipsnames,svgnames,x11names}{xcolor}
%
\documentclass[
  letterpaper,
  DIV=11,
  numbers=noendperiod,
  oneside]{scrartcl}

\usepackage{amsmath,amssymb}
\usepackage{iftex}
\ifPDFTeX
  \usepackage[T1]{fontenc}
  \usepackage[utf8]{inputenc}
  \usepackage{textcomp} % provide euro and other symbols
\else % if luatex or xetex
  \usepackage{unicode-math}
  \defaultfontfeatures{Scale=MatchLowercase}
  \defaultfontfeatures[\rmfamily]{Ligatures=TeX,Scale=1}
\fi
\usepackage{lmodern}
\ifPDFTeX\else  
    % xetex/luatex font selection
\fi
% Use upquote if available, for straight quotes in verbatim environments
\IfFileExists{upquote.sty}{\usepackage{upquote}}{}
\IfFileExists{microtype.sty}{% use microtype if available
  \usepackage[]{microtype}
  \UseMicrotypeSet[protrusion]{basicmath} % disable protrusion for tt fonts
}{}
\makeatletter
\@ifundefined{KOMAClassName}{% if non-KOMA class
  \IfFileExists{parskip.sty}{%
    \usepackage{parskip}
  }{% else
    \setlength{\parindent}{0pt}
    \setlength{\parskip}{6pt plus 2pt minus 1pt}}
}{% if KOMA class
  \KOMAoptions{parskip=half}}
\makeatother
\usepackage{xcolor}
\usepackage[left=1in,marginparwidth=2.0666666666667in,textwidth=4.1333333333333in,marginparsep=0.3in]{geometry}
\setlength{\emergencystretch}{3em} % prevent overfull lines
\setcounter{secnumdepth}{-\maxdimen} % remove section numbering
% Make \paragraph and \subparagraph free-standing
\makeatletter
\ifx\paragraph\undefined\else
  \let\oldparagraph\paragraph
  \renewcommand{\paragraph}{
    \@ifstar
      \xxxParagraphStar
      \xxxParagraphNoStar
  }
  \newcommand{\xxxParagraphStar}[1]{\oldparagraph*{#1}\mbox{}}
  \newcommand{\xxxParagraphNoStar}[1]{\oldparagraph{#1}\mbox{}}
\fi
\ifx\subparagraph\undefined\else
  \let\oldsubparagraph\subparagraph
  \renewcommand{\subparagraph}{
    \@ifstar
      \xxxSubParagraphStar
      \xxxSubParagraphNoStar
  }
  \newcommand{\xxxSubParagraphStar}[1]{\oldsubparagraph*{#1}\mbox{}}
  \newcommand{\xxxSubParagraphNoStar}[1]{\oldsubparagraph{#1}\mbox{}}
\fi
\makeatother


\providecommand{\tightlist}{%
  \setlength{\itemsep}{0pt}\setlength{\parskip}{0pt}}\usepackage{longtable,booktabs,array}
\usepackage{calc} % for calculating minipage widths
% Correct order of tables after \paragraph or \subparagraph
\usepackage{etoolbox}
\makeatletter
\patchcmd\longtable{\par}{\if@noskipsec\mbox{}\fi\par}{}{}
\makeatother
% Allow footnotes in longtable head/foot
\IfFileExists{footnotehyper.sty}{\usepackage{footnotehyper}}{\usepackage{footnote}}
\makesavenoteenv{longtable}
\usepackage{graphicx}
\makeatletter
\def\maxwidth{\ifdim\Gin@nat@width>\linewidth\linewidth\else\Gin@nat@width\fi}
\def\maxheight{\ifdim\Gin@nat@height>\textheight\textheight\else\Gin@nat@height\fi}
\makeatother
% Scale images if necessary, so that they will not overflow the page
% margins by default, and it is still possible to overwrite the defaults
% using explicit options in \includegraphics[width, height, ...]{}
\setkeys{Gin}{width=\maxwidth,height=\maxheight,keepaspectratio}
% Set default figure placement to htbp
\makeatletter
\def\fps@figure{htbp}
\makeatother

\usepackage{booktabs}
\usepackage{longtable}
\usepackage{array}
\usepackage{multirow}
\usepackage{wrapfig}
\usepackage{float}
\usepackage{colortbl}
\usepackage{pdflscape}
\usepackage{tabu}
\usepackage{threeparttable}
\usepackage{threeparttablex}
\usepackage[normalem]{ulem}
\usepackage{makecell}
\usepackage{xcolor}
\KOMAoption{captions}{tableheading}
\makeatletter
\@ifpackageloaded{caption}{}{\usepackage{caption}}
\AtBeginDocument{%
\ifdefined\contentsname
  \renewcommand*\contentsname{Table of contents}
\else
  \newcommand\contentsname{Table of contents}
\fi
\ifdefined\listfigurename
  \renewcommand*\listfigurename{List of Figures}
\else
  \newcommand\listfigurename{List of Figures}
\fi
\ifdefined\listtablename
  \renewcommand*\listtablename{List of Tables}
\else
  \newcommand\listtablename{List of Tables}
\fi
\ifdefined\figurename
  \renewcommand*\figurename{Figure}
\else
  \newcommand\figurename{Figure}
\fi
\ifdefined\tablename
  \renewcommand*\tablename{Table}
\else
  \newcommand\tablename{Table}
\fi
}
\@ifpackageloaded{float}{}{\usepackage{float}}
\floatstyle{ruled}
\@ifundefined{c@chapter}{\newfloat{codelisting}{h}{lop}}{\newfloat{codelisting}{h}{lop}[chapter]}
\floatname{codelisting}{Listing}
\newcommand*\listoflistings{\listof{codelisting}{List of Listings}}
\makeatother
\makeatletter
\makeatother
\makeatletter
\@ifpackageloaded{caption}{}{\usepackage{caption}}
\@ifpackageloaded{subcaption}{}{\usepackage{subcaption}}
\makeatother
\makeatletter
\@ifpackageloaded{sidenotes}{}{\usepackage{sidenotes}}
\@ifpackageloaded{marginnote}{}{\usepackage{marginnote}}
\makeatother

\ifLuaTeX
  \usepackage{selnolig}  % disable illegal ligatures
\fi
\usepackage{bookmark}

\IfFileExists{xurl.sty}{\usepackage{xurl}}{} % add URL line breaks if available
\urlstyle{same} % disable monospaced font for URLs
\hypersetup{
  pdftitle={Prediction Interval and its Callibration},
  pdfauthor={Shishir Rao},
  colorlinks=true,
  linkcolor={blue},
  filecolor={Maroon},
  citecolor={Blue},
  urlcolor={Blue},
  pdfcreator={LaTeX via pandoc}}


\title{Prediction Interval and its Callibration}
\author{Shishir Rao}
\date{}

\begin{document}
\maketitle


\section{Introduction}\label{introduction}

A potential customer would like to know how long will an expensive piece
of industrial equipment they intend to purchase operate before failing.
A maintenance manager needs to make a decision on whether an operating
unit should be shut down and brought in for maintenance or let it run
for some more time. A manufacturing company would like to allocate
budget for potential warranty claim returns and would like to know the
number of product failures in the next 6 months.

Questions of the above nature can be addressed by analyzing data from
past failures and constructing prediction intervals. In this blog
article, we will discuss prediction intervals, how are they different
from confidence intervals and why is it necessary to calibrate
prediction intervals. The methods discussed in this blog post are from
the chapter on prediction intervals in a textbook\sidenote{\footnotesize Statistical
  Methods for Reliability Data, Second Edition (William Q. Meeker, Luis
  A. Escobar, Francis G. Pascual)} that I have been reading. I highly
recommend this book to anyone interested in applying statistical methods
in the field of reliability engineering.

\section{Prediction Intervals vs Confidence
Intervals}\label{prediction-intervals-vs-confidence-intervals}

A prediction interval is an interval within which a future observation
is likely to fall, whereas a confidence interval is an interval within
which a population parameter is likely to fall. Lets look at this
difference with the help of an example.

Suppose there are 1000 houses that are on sale in a particular
neighborhood. We are interested in finding out the average sale price of
a house in this neighborhood. If we had access to the 1000 sale prices,
finding the average is a straightforward exercise. Now suppose that only
50 have been sold so far and we want to use this data to estimate the
average sale price of the 1000 houses in the neighborhood.

In the above example, our sample size is 50 and the population size is
1000. The population parameter that we want to estimate is the mean (or
average) of 1000 homes \sidenote{\footnotesize Other quantities like the median or any
  other quantile are also population parameters}. This population
parameter is an unknown quantity, but it is fixed!\sidenote{\footnotesize We are
  talking about the frequentist approach. In the Bayesian approach,
  parameters are also treated as random quantities.} We can use the 50
sale prices to construct a confidence interval for the mean of 1000 sale
prices.

On the other hand, suppose we want to know the price range within which
the next house that sells is going to fall in. This range would be the
prediction interval. Note that the sale price of the next house that is
sold is a random quantity, unlike the unknown population parameter which
is fixed. The prediction that we are interested in is an ``unknown and
random'' quantity whereas the average sale price of a house in the
neighborhood is an ``unknown but fixed'' quantity. Confidence intervals
are constructed for ``unknown but fixed'' population
parameters\sidenote{\footnotesize Like the mean in this case} and prediction intervals
are constructed for the ``unknown and random'' quantity\sidenote{\footnotesize Sale
  price of the next house sold in this case.}

\section{Reliability Applications}\label{reliability-applications}

Similar to the example of house sale prices, consider a scenario where a
customer wants to purchase an expensive piece of industrial equipment
and is interested in the prediction interval for that particular
equipment. The confidence interval would give the customer an idea of
the mean time to failure of all identical equipment manufactured in the
past, present and future by the manufacturer. Although this is good
information to have, the customer wants an interval for one equipment.
This interval, which is the prediction interval, accounts for individual
variability in the time-to-failure.

We can still use data from historical failures to get a prediction
interval for an equipment. The difference between this case and the
previous example of house prices is that historical data from equipment
failures may be incomplete. This means that in addition to failures, the
data might consist of censored observations, where the equipment hasn't
failed by the end of the observation period\sidenote{\footnotesize This is for right
  censored observations. We can also have left or interval censored
  observation and even truncated observations in reliability data.}.

In the following article, a dataset consisting of past failures (and
censored) observations will be used to construct a prediction interval.
It will then be calibrated to ensure adequate coverage
probability\sidenote{\footnotesize Will be discussed later in the article}. All
computations are performed using R.

\section{Case Study: Mechanical
Switches}\label{case-study-mechanical-switches}

\subsection{Data}\label{data}

Table 1 shows failure times of 40 randomly selected mechanical switches
tested in a facility. 3 switches had not failed by the end of the test,
leading to right censored data.

There are 2 modes of failure - Spring A and Spring B. We are interested
in constructing a prediction interval for a mechanical switch
irrespective of mode of failure\sidenote{\footnotesize The 2 modes of failure are
  competing risks, but this information is not considered in this
  analysis.}.

The structure of the rest of the articles is as follows

\subsection{Quarto}\label{quarto}

Quarto enables you to weave together content and executable code into a
finished document. To learn more about Quarto see
\url{https://quarto.org}.

\subsection{Running Code}\label{running-code}

When you click the \textbf{Render} button a document will be generated
that includes both content and the output of embedded code. You can
embed code like this:

You can add options to executable code like this

The \texttt{echo:\ false} option disables the printing of code (only
output is displayed).




\end{document}
